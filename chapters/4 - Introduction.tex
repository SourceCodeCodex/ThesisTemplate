\chapter{Introducere}
\thispagestyle{pagestyle}

Fiecare capitol trebuie să aibă o structură clară, va începe pe pagină nouă și va conține un titlu. Va fi urmat de două linii de 12 pt lăsate libere.

\section{Informații generale}

Fiecare secțiune a unui capitol (ex. 1.1 INFORMAȚII GENERALE) va fi poziționat la un rând liber sub text și va avea un rând liber de12 pt deasupra textului. 

Textul lucrării va fi aliniat uniform (justify). Este de preferat ca textul să fie verificat pentru eventualele erori în limba de editare cu ajutorul facilității de verificare a ortografiei (speller) din programul word. Este recomandat ca lucrarea de finalizare a studiilor să nu depășească 100 de pagini, inclusiv anexele.

Reguli aplicate pentru textul lucrării:
\begin{enumerate}[leftmargin=2cm,topsep=1.15pt,itemsep=1.15pt,partopsep=1.15pt,parsep=1.15pt,label=\alph*.]
   \item Marginile paginii – se vor utiliza următoarele valori pentru marginile paginii
   \begin{itemize}[topsep=1.15pt,itemsep=1.15pt,partopsep=1.15pt,parsep=1.15pt]
     \item interior: 2 cm 
     \item exterior: 2 cm 
     \item sus: 2,5 cm (inclusiv header)
     \item jos: 2 cm
   \end{itemize}
   \item Spațiere între rânduri - textul va respecta o spațiere între rânduri de 1,15 linii
   \item Alinierea textului în cadrul paragrafelor - textul din cadrul paragrafelor normale va fi aliniat între marginile din stânga şi dreapta. Primul rând al fiecărui paragraf va avea o aliniere de 1,5 cm. Excepție fac titlurile capitolelor, care vor fi aliniate la stânga, precum și etichetele tabelelor și ale figurilor (conform explicațiilor de mai jos)
   \item Font – fontul utilizat pentru redactare va fi Arial, cu dimensiunea de 12 puncte, utilizând diacriticele specifice limbii în care este redactată lucrarea (ex: ă, ş, ţ, î, â - pentru limba română);
   \item Numerotarea paginilor - numerotarea paginilor se face începând cu pagina de titlu, până la ultima pagină a lucrării, dar numărul paginii apare doar începând cu Introducerea. Numărul de pagină se inserează în subsolul paginii, centrat.
   \item Antetul paginii – apare începând cu introducerea și va conține pe rânduri succesive un text cu înălțimea de 8, aliniat la stânga: (i) textul Universitatea Politehnica Timișoara ; (ii) denumirea programului de studii și anul susținerii ; (iii) numele candidatului (în stânga) și titlul lucrării. În partea dreaptă a antetului poate fi integrată sigla UPT;
\end{enumerate}