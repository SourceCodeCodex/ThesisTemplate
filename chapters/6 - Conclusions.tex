\chapter{Concluzii}
\thispagestyle{pagestyle}

Lucrarea se va încheia cu un capitol de concluzii. Acesta va conţine principalele rezultate ale lucrării şi implicaţiile practice ale acestora. În cazul proiectelor de diplomă, se vor menționa principalele date sintetice obținute din procesul de proiectare.

\section{Bibliografie}
La sfârşitul lucrării va fi dată o listă de referinţe pentru textele ştiinţifice consultate pe parcursul realizării lucrării. Vor fi trecute toate sursele, inclusiv cele de pe internet. Acestea vor fi referite în text şi trecute în lista de referinţe în ordine alfabetică, după exemplele de mai jos. 

Bibliografia trebuie să cuprindă toate titlurile din literatura de specialitate care au servit ca bază de documentare, respectiv autorii care au fost citați în text, la toate capitolele lucrării. 

În cadrul Facultății de Automatică și Calculatoare se cere folosirea stilului de citare IEEE (detalii IEEE Citation Guidelines2.doc (ieee-dataport.org)), folosit cu precădere în publicațiile științifice din domeniul IT. Cele trei părți importante ale referinței sunt:
\begin{enumerate}[leftmargin=2cm,topsep=1.15pt,itemsep=1.15pt,partopsep=1.15pt,parsep=1.15pt,label=\alph*.]
   \item Numele autorului indicat ca prima inițială a prenumelui, apoi numele complet.
   \item Titlul articolului, brevetul, lucrarea de conferință etc., între ghilimele.
   \item Titlul revistei sau cărții cu caractere cursive
Modul de redactare a referinței depinde de tipul publicației, vă rugăm să urmăriți cu atenție indicațiile de la link-ul de mai sus.
\end{enumerate}
	 
Fiecare citare trebuie notă în text prin utilizarea unor numere secvențiale simple. Un număr cuprins între paranteze drepte, plasat în textul raportului, indică referința specifică. Citările sunt numerotate în ordinea în care apar. Odată ce o sursă a fost citată, același număr este folosit în toate referințele ulterioare din text. Nu se face distincție între sursele electronice și cele tipărite, cu excepția detaliilor referințelor citate.

Fiecare număr de referință trebuie să fie cuprins între paranteze drepte pe aceeași linie cu textul, înaintea oricărei semne de punctuație, cu un spațiu înaintea parantezei.

Exemple:
\begin{enumerate}[leftmargin=2cm,topsep=1.15pt,itemsep=1.15pt,partopsep=1.15pt,parsep=1.15pt,label=\alph*.]
   \item ". . .finalul cercetării mele [13]."
   \item "Teoria a fost prezentată pentru prima dată în 1987 [1]."
\end{enumerate}

Lista de referințe din bibliografie este compusă din toate sursele folosite pentru documentarea lucrării și se realizează în ordinea numerică a citării în text și nu în ordine alfabetică a autorilor.

Preluarea identică a unei fraze sau paragraf va fi citată prin indicarea inclusiv a paginii din sursa utilizată, dar și prin ghilimele şi forma italică a literelor; pentru sursele preluate de pe internet, vor fi notate adresele de pagină web; în lista bibliografică finală lucrările se trec în ordinea alfabetică a numelor autorilor. La lucrările colective, regula referitoare la ordinea alfabetică este valabilă pentru primul autor. 

Dacă se citează site-uri web, reviste sau articole, înainte de acestea se vor trece trei asteriscuri, informații referitoare la volum, număr, pagini consultate, adresa web exactă a articolului respectiv, data vizitării site-ului și a descărcării materialului, data accesării. Adresele de pagini web se regăsesc la finalul listei.

Sursele bibliografice la care nu se poate menționa autorul se vor specifica astfel: "***" urmat de denumirea articolului și/sau a cărții, editura și locul apariției (pentru cărți), volumul, numărul acestuia, prima și ultima pagină a lucrării citate, anul apariției. 

*** https://ro.wikipedia.org/wiki/Motor accesare februarie 2022

Exemplu citare: Einstein \cite{einstein}

\section{Declarația de autenticitate}
Ultima pagină a lucrării de licență/diplomă/disertație trebuie să conțină "Declarația de originalitate a lucrării de finalizare a studiilor", completată olograf, în conformitate cu cerinţele UPT. Declaraţia se descarcă de pe adresa de web: 

\url{http://www.upt.ro/img/files/Regulamente_UPT/2020/Declaratie_de_autenticitate_UPT_2020.pdf}